

\section{Session 5: Compliance gradient computation}
\begin{center}
Albert Torres Rubio \\
31/10/2017
\end{center}

\subsection{Problem Definition}
In order to solve a topological optimization problem the gradient of the cost function must be computed, performing this calculation in order to minimize the compliance (as an example), the following steps must be executed to find out which is the gradient of the function.
Firstly, defining:
$$ l(v) = \int_{\Gamma_N} t \cdot v $$
$$ a(\rho, u, v)= \int_\Omega J' u : C(\rho) : \nabla^s u $$

In fact, the initial problem we have is the following:

$$ \min \: l(u) $$
subject to

$$ \left.
\begin{array}{c}
\int_\Omega \rho \leq V \\
a(\rho, u, v)= l(v)  \\
\end{array} \right\}     \quad \forall  v \in \gamma \quad and \quad \forall \rho \in \chi$$
Where,

$$ \begin{array}{c}
\gamma = \{ \phi \in H^{1} (\rho) : \phi \mid_{\Gamma_{0} }  = 0 \} \\
\chi = L^{\infty
} (\Omega, \{0,1\})  \\
\end{array}$$

\subsection{Differential definition}
Performing the calculation of the differential of the cost function:
$$l(u(\rho + \tilde{\rho}))-l(u(\rho))=l[u(\rho + \tilde{\rho})-u(\rho)]=$$
$$l[D_\rho u(\rho)\tilde{\rho} + O(\rho^2)]= l[D_\rho u(\rho) \tilde{\rho}]+O(\rho^2)$$
Therefore,

\begin{equation} \label{E:D}
Dl(\rho)\tilde{\rho} =  l[D_\rho u(\rho) \tilde{\rho}]
\end{equation}
Where $D_\rho$ is the derivative respect to $\rho$.

\subsection{Gradient Computation}
Defining,

$$F(\rho)=a(\rho, u(\rho),v)-l(v)=0 \qquad \forall v, \: \forall \rho$$

Therefore, computing the differential:

$$\underbrace{F(\rho+\tilde{\rho})}_\text{=0}-\underbrace{F(\rho)}_\text{=0}=a(\rho+\tilde{\rho},u(\rho+\tilde{\rho}),v)-a(\rho,u(\rho),v) =$$
$$a(\rho+\tilde{\rho},u(\rho+\tilde{\rho}),v)\textcolor{red}{ \: -\: a(\rho+\tilde{\rho},u(\rho),v)}\textcolor{red}{ \: +\: a(\rho+\tilde{\rho},u(\rho),v)} - a(\rho,u(\rho),v)=$$
$$a(\rho+\tilde{\rho},Du(\rho)\tilde{\rho},v)+D_\rho a(\rho,u(\rho),v)\tilde{\rho}+O(\tilde{\rho}^2)=$$
$$a(\rho,Du(\rho)\tilde{\rho},v)+D_\rho a(\rho,u(\rho),v)\tilde{\rho}+O(\tilde{\rho}^2)=0$$

Which lasts in

\begin{equation} \label{E:a}
a(\rho,Du(\rho)\tilde{\rho},v) = \boxed{-D_\rho a(\rho,u(\rho),v) \tilde{\rho}} + O(\tilde{\rho}^2)  \qquad  \forall  v \in \gamma, \forall \rho \in \mathbb{R}
\end{equation}
Continuing with Equation \ref{E:D}:
$$l(u(\rho + \tilde{\rho}))-l(u(\rho))=l[D_\rho u(\rho) \tilde{\rho}] + O(\tilde{\rho}^2) = a(\rho,u(\rho),D_\rho u(\rho)\tilde{\rho})+ O(\tilde{\rho}^2)=$$
$$a(\rho,D_\rho u(\rho)\tilde{\rho},u(\rho))+ O(\tilde{\rho}^2) =-D_\rho a(\rho,u(\rho)\tilde{\rho},u(\rho))+ O(\tilde{\rho}^2)=$$
$$-\left[a(\rho+\tilde{\rho}, u(\rho),v)-a(\rho, u(\rho),v) \right] + O(\tilde{\rho}^2)=$$
$$-\int_{\Omega} \nabla^s u :[C(\rho+\tilde{\rho})-C(\rho)]:\nabla^s u + O(\tilde{\rho}^2)=$$
$$-\int_{\Omega} \nabla^s u : C'(\rho)\tilde{\rho}:\nabla^s u + O(\tilde{\rho}^2)=$$
$$=Dl(\rho)\tilde{\rho} + O(\rho^2)$$


Therefore,
$$Dl(\rho)\tilde{\rho} = \int_{\Omega} \underbrace{(\nabla^s u : C'(\rho):\nabla^s u)}_\text{g} \tilde{\rho}= (g,\tilde{\rho})_{L^2}$$
Finally, the computed gradient is:

\begin{equation}
\boxed{g=\nabla^s u : C'(\rho):\nabla^s u)}
\end{equation}

Where,

$$C(\rho)=2\mu(\rho)I+[\kappa(\rho)-\mu(\rho)] I\otimes I$$

\clearpage
























\section{Session 1: Topology optimization Problem}

\subsection{Formulation of the topology optimization problem}
Let $ \Omega $ be the domain of the problem and let $\Omega^+ $ and $ \Omega^- $ be a partion that represents strong and weak materials respectively, such that:
\begin{equation}
\Omega=\Omega^+\cup\Omega^-
\end{equation}
The fundamental mathematical statement of a topology optimization
problem is defined through the design variable, the objective function and a set of constraints. Based on linear static finite element discretization, the standard topology optimization problem may be given as: 
\begin{equation}
P1:\left\{ \begin{array}{cc}
\min & f(\chi,U)\\
x, u & \\
s.t. & K(\chi)U=F(\chi)\\
 & g_{i}(\chi,U)\leq0
\end{array}\right.\label{eq:problem topology optimization}
\end{equation}
where $f$ is the objective function, $\chi$ is the vector of integer
nodal design variables (solid or void), $U$ is the displacement vector,
$K$ is the global stiffness matrix, $F$ is the force vector and
$g_{i}$ are the constraints. The characteristic function $\chi$ is defined as follows:
\begin{equation}
\chi(x)=\left\{ \begin{array}{cc}
1 & x \in \Omega^+ \\
0 & x \in \Omega^-
\end{array}\right.\label{eq:chi_def}
\end{equation}
Where $x$ is the position of each point of the domain. Within this generalized statement, a number of problems can be formulated
considering a variety of objectives and constraints, including compliance,
stresses, volume, perimeter and inverse problems. As an example, the
popular compliance problem can be setup by minimizing an objective
of structural compliance as $f=U^{T}KU$ and constraining the amount
of material usage as $g=\frac{\int \chi}{V_{0}}-V^{*}\le0$.\\\\
As defined in Eq. \ref{eq:problem topology optimization}, the stiffness matrix is written in terms of the characteristic function $\chi$.
\begin{equation}
K=\int_{\Omega}^{}B^TCBd\Omega=\int_{\Omega^+}^{}B^TC^+Bd\Omega+\int_{\Omega^-}^{}B^TC^-Bd\Omega
\end{equation}
Which, if expressed over the whole volume, gives the following expression:
\begin{equation}
K=\int_{\Omega}^{}\chi B^TC^+Bd\Omega+(1-\chi)B^TC^-Bd\Omega
\label{eq:stiffness_chi}
\end{equation}
Therefore, if comparing Eq. \ref{eq:stiffness_chi} with the definition of the stiffness matrix, the elasticity matrix $ C $ can be written in terms of the characteristic function $\chi$, by means of $C^+$ and $C^-$:
\begin{equation}
C=\chi C^+ + (1-\chi)C^-
\end{equation}


\subsection{Regularization of the problem}

Fundamentally, the formulation in (\ref{eq:problem topology optimization})
poses an extremely challenging large-scale integer programming problem.
As a result, it is desirable to replace the integer variable with
continuous variables and identify a means to iteratively steer the
solution towards a discrete solid/void solution. This is accomplished
with a regularization function. Finally, material interpolation
schemes are then used characterize the material in grey areas.

A general formulation of the regularized problem based on linear static
finite element analysis may be given as: 

\begin{equation}
P2:\left\{ \begin{array}{cc}
\min & f(\rho,U)\\
\rho, u & \\
s.t. & K(\rho)U=F(\rho)\\
 & g_{i}(\rho,U)\leq0
\end{array}\right.\label{eq:problem topology optimization-1}
\end{equation}

where $f$ is the objective function, $\rho$ is the vector of nodal
densities, $\rho \in \left[0, 1\right]$, $U$ is the displacement vector, $K$ is the global stiffness
matrix, $F$ is the force vector and $g_{i}$ are the constraints.

\subsection{Interpolation Scheme}
At this point, instead of expressing the domain by means of an integer function $ \chi $, the domain is expressed by a continuous function $\rho$. However, there has to be a way to interpret the physical meaning of $\rho$. For instance, if $ \chi=1 $ represents a point in the domain where there is material, and  $ \chi=0 $ represents a point in the domain where there is void, then the interpretation of an intermediate value of $ \rho=0.75 $ has to be determined. Interpolation schemes will allow to establish a relation between the function $\rho$ and its corresponding constitutive tensor $C(\rho)$. 
\subsubsection{SIMP}

SIMP mathematical formulation is presented in the following equation:
\begin{equation}
C(\rho)=\rho^{p}C^{+}+(1-\rho^{p})C^{-}\label{eq:SIMP formulation-1}
\end{equation}
Where $C$ is the elasticity matrix ($C^{+}$ for solid material and
$C^{-}$ for void), $\rho$ is the regularized characteristic function and $p$ is any value higher
than 1. It has been showed that the a proper value for this parameter is $ p=3 $. In order to satisfy the Hashin-Shtrikman (HS) bounds for two-phase
materials, and following the reference book, the penalization factor is computed according to (\ref{eq:p function nu SIMP-1}).
\begin{equation}
p(\nu^{+})=\max\left\{ \frac{2}{1-\nu^{+}},\frac{4}{1+\nu^{+}}\right\} \label{eq:p function nu SIMP-1}
\end{equation}
Note that the elasticity matrix can be expressed as a function of the Young modulus and the Poisson ratio, i.e,
\begin{equation}
C^-=\frac{E^-}{1-2\nu^-}\left(\begin{matrix} 
1 & \nu^- & 0 \\
\nu^- & 1 & 0 \\
0 & 0 & \frac{1-(v^-)^2}{2} 
\end{matrix}\right)
\hspace{1cm}
C^+=\frac{E^+}{1-2\nu^+}\left(\begin{matrix} 
1 & \nu^+ & 0 \\
\nu^+ & 1 & 0 \\
0 & 0 & \frac{1-(v^+)^2}{2} 
\end{matrix}\right)
\end{equation}
Inserting it into to Eq. (\ref{eq:SIMP formulation-1}):
\begin{equation}
	C(\rho)=(\rho^pE^++(1-\rho^p)E^-)\frac{1}{1-2\nu}\left(\begin{matrix}
		1     & \nu^+ & 0                   \\
		\nu^+ & 1     & 0                   \\
		0     & 0     & \frac{1-(v^+)^2}{2}
	\end{matrix}\right)=E(\rho)\frac{1}{1-2\nu}\left(\begin{matrix} 
	1 & \nu^+ & 0 \\
	\nu^+ & 1 & 0 \\
	0 & 0 & \frac{1-(v^+)^2}{2} 
	\end{matrix}\right)
\end{equation}
Where $ \nu^-=\nu^+=\nu $ has been assumed constant for both $ C^+ $ and $ C^- $. Now, an expression for the Young modulus in terms of $\rho$ and $p$ has been obtained. In isotropic materials, the shear and bulk modulus, $ \kappa $ and $ \mu $, can be obtained by means of $ E $ and $ \nu $, which are used to computed the Hashin-Strikman bounds.

\paragraph{Hashin-Shtrikman bounds}
Hashin-Shtrikman  bounds are the tightest bounds possible from range of composite moduli for a two-phase material, in the form of matrix and inclusion.
On the one hand, the upper bound is derived by considering the solid
material ($\mu^{+}$, $\kappa^{+}$) as the matrix, and the void material
($\mu^{-}$, $\kappa^{-}$) is the inclusion in the form of a sphere
(or circle in 2D). On the other hand, the lower bound is derived from
the opposite assumption, where the void material ($\mu^{-}$, $\kappa^{-}$)
is the matrix and the solid material ($\mu^{+}$, $\kappa^{+}$) the
inclusion. A visual representation of the concept is presented in
Figure \ref{fig:Visual-representation-of}. And kind of interpolation must be in between the upper and lower band in order to physically understand the meaning of a grey material.
\begin{figure}[H]
\begin{centering}
\begin{tabular}{>{\centering}m{7cm}>{\centering}m{7cm}}
\includegraphics[width=6.5cm]{\string"img/28a - HS bounds\string".png} & \centering{}\includegraphics[width=6.5cm]{\string"img/28b - HS bounds interpolation\string".PNG}\tabularnewline
\end{tabular}
\par\end{centering}
\caption{Visual representation of the Hashin-Shtrikman bounds\label{fig:Visual-representation-of}}
\end{figure}

The Hashin-Shtrikman bounds are the limits of the all the possible micro-structures that, for instance, a point material with a $ \rho=0.75 $ may have. 
